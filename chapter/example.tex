% chapter Example
%------------------------------------------------
\chapter{Example}

%------------------------------------------------
\section{Figure}

\begin{figure}
  \centering
  \includegraphics[width=0.6\textwidth]{latex-featured}
  \caption{Example Figure.}
  \label{fig:examplefigure}
\end{figure}

%------------------------------------------------
\section{Table}

\begin{table}
	\centering
	\setstretch{1.5}
	\caption{Example Table.}
	\label{tab:exampletable}
	\begin{tabular}{ccc}
		\toprule
		Name & Age & Location \\ \midrule
    John & 24 & DE \\
    Gabriela & 22 & MY \\
    Michael & 21 & US \\ \bottomrule
	\end{tabular}
\end{table}

%------------------------------------------------
\section{Equation}

Example Equation.
\begin{equation}
	\int_{a}^{b} f(x) \, dx\ \approx \frac{b - a}{2} \cdot [f(a) + f(b)]
	\label{eq:exampleequation}
\end{equation}

%------------------------------------------------
\section{Label}

Example label referring to \autoref{fig:examplefigure}, \autoref{tab:exampletable}, and \autoref{eq:exampleequation}.

%------------------------------------------------
\section{Acronym}
Example Acronym of \ac{WWW}.

%------------------------------------------------
\section{Citation}

Example citation \cite{shannon1949communication}.

%------------------------------------------------
\section{Math Mode}

% \( ... \) is LaTeX syntax. $ ... $ is TeX syntax.
% plainTeX only allows $.
% In LaTeX you can use both, but \( ... \) will give less obscure error messages when there is a mistake inside it.
% https://tex.stackexchange.com/questions/510/are-and-preferable-to-dollar-signs-for-math-mode/513#513

Example of math equation inline (within paragraph) $ \int_{0}^{10} x^2 $ with some text at the end.

Another example math equation inline (within paragraph) \( \int_{0}^{10} x^2 \) with some text at the end.

Example of math equation outside the paragraph \[ \int_{0}^{10} x^2 \] with some text at the end.

