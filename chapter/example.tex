% chapter Example
%------------------------------------------------
\chapter{Example}

%------------------------------------------------
\section{Figure}

\begin{figure}
  \centering
  \includegraphics[width=0.6\textwidth]{latex-featured}
  \caption{Example Figure.}
  \label{fig:examplefigure}
\end{figure}

\begin{sidewaysfigure}
  \centering
  \includegraphics[width=0.8\textwidth]{latex-featured}
  \caption{Example Sideways Figure.}
  \label{fig:examplesidewaysfigure}
\end{sidewaysfigure}

%------------------------------------------------
\section{Table}

\begin{table}
  \centering
  \setstretch{1.5}
  \caption{Example Table.}
  \label{tab:exampletable}
  \begin{tabular}{ccc}
    \toprule
    Name & Age & Location \\ \midrule
    John & 24 & DE \\
    Gabriela & 22 & MY \\
    Michael & 21 & US \\ \bottomrule
  \end{tabular}
\end{table}

%------------------------------------------------
\section{Equation}

Example Equation.
\begin{equation}
  \int_{a}^{b} f(x) \, dx\ \approx \frac{b - a}{2} \cdot [f(a) + f(b)]
  \label{eq:exampleequation}
\end{equation}


%------------------------------------------------
\section{Algorithm}

Example Algorithm.

\begin{algorithm}
  \centering
  \caption{Algorithm with caption}
  \label{alg:examplealgorithm}
  \begin{algorithmic}
  \State $i \gets 10$
  \If{$i\geq 5$}
    \State $i \gets i-1$
  \Else
    \If{$i\leq 3$}
      \State $i \gets i+2$
    \EndIf
  \EndIf
  \end{algorithmic}
\end{algorithm}

%------------------------------------------------
\section{Label}

Example label referring to \autoref{fig:examplefigure}, \autoref{tab:exampletable},
\autoref{alg:examplealgorithm} and \autoref{eq:exampleequation}.

%------------------------------------------------
\section{Acronym}
Example Acronym of \ac{WWW}.

%------------------------------------------------
\section{Citation}

Example citation \cite{shannon1949communication}.

%------------------------------------------------
\section{Math Mode}

% \( ... \) is LaTeX syntax. $ ... $ is TeX syntax.
% plainTeX only allows $.
% In LaTeX you can use both, but \( ... \) will give less obscure error messages when there is a mistake inside it.
% https://tex.stackexchange.com/questions/510/are-and-preferable-to-dollar-signs-for-math-mode/513#513

Example of math equation inline (within paragraph) $ \int_{0}^{10} x^2 $ with some text at the end.

Another example math equation inline (within paragraph) \( \int_{0}^{10} x^2 \) with some text at the end.

Example of math equation outside the paragraph \[ \int_{0}^{10} x^2 \] with some text at the end.

%------------------------------------------------
\section{Quantity: Number and Unit}
% siunitx package

Example of quantity \qty{10}{\kg\m\per\s\squared} or \qty[per-mode=symbol]{1.5}{\kJ\per\mol}.

Example of quantity list \qtylist{10;20;30}{\uA}.

Example of quantity range \qtyrange{10}{20}{\GHz}.

Example of quantity product \qtyproduct{10x20x30}{\cm}

Example of angle \ang{10.5} and \ang{10;30}.

% Number must have equal amount of decimal places.
Example of uncertainty \qty{100.0(10.5)}{\nF} or \qty[separate-uncertainty-units = repeat]{200.00(20.55)}{\MW}.

Example of plus-minus sign \qty{\pm 0.5}{\ohm}.

